\documentclass{beamer}
\mode<presentation>
{
  \setbeamertemplate{background canvas}[vertical shading][bottom=orange!10,top=blue!10]

  \usetheme{JuanLesPins}
  \usefonttheme[onlysmall]{structurebold}
}
\usepackage{multirow}
\title{Bayesian Methods for Inference on the Causal Effects of Mediation}
\author[]{Chanmin Kim}
\date{Final Defense \\ June 26 2013}
\institute{Department of Statistics \\ Univsesity of Florida}

\begin{document}
\begin{frame}
 \titlepage
\end{frame}
\begin{frame}
 \frametitle{Outline}\section{}

\tableofcontents[currentsubsection]

\end{frame}

\section{Introduction}
\begin{frame}
\frametitle{What is the Causal Effect of Mediation?}
\begin{itemize}
\item Investigators interest in {\color{red} Causal Mechanism} as well as causal effects.
\item Randomized experiments often only determine the effect of the
  treatment on the outcome.
\item Causal mediation analysis:
\begin{equation*}
\begin{array}{ccccc}
  &          & \color{red} Mediator\, (M) & &\\
  & \color{red} \nearrow & & \color{red}\searrow & \\
Treatment\, (Z) & & \longrightarrow & & Outcome\, (Y) 
\end{array}
\end{equation*}
\item Question : How can we learn about the causal mechanism from
  observational or experimental studies?
\end{itemize}
\end{frame}

\begin{frame}
\frametitle{Examples}
\begin{itemize}
\item Family and social support mediating the effect of care
  managed-based interventions on depression.
\item The effect of vaccines can be from stimulation of subject's
  immune system or reducing risk of infection (herd effect) among
  population.
\item Motivation and self-efficacy mediating the effect of perceived
  person-job fit on job performance. 
\item Compliance level (measured by self-monitoring records) mediating
  the effect of face-to-face counseling treatment on weight loss. 
\end{itemize}
\end{frame}



\begin{frame}
\frametitle{Potential Outcomes Framework}
Also called `Rubin Causal Model' (RCM) (Holland, 1986)
\begin{itemize}
\item Binary treatment :   $Z$ ( Control=0 vs. Intervention=1).
\item Mediator : $M$
\item Outcome : $Y$
\item Baseline covariates : $X$
\item Potential mediators : $M_z$ where $M_{obs}=Z M_1 + (1-Z) M_0$ observed
\item {\color{red} Potential outcomes
} : $Y_{z,m}$ where $Y_{obs}=Z Y_{1,M_{1}} + (1-Z) Y_{0,M_0}$
  observed
\end{itemize}
Only one of $\{Y_{1,M_1},Y_{0,M_0}\}$ can be observed for each
$i$. And $\{Y_{1,M_0},Y_{0,M_1}\}$ are never observed!
\end{frame}

\begin{frame}
\frametitle{Causal Mediation Effects}
Then,
\begin{itemize}
\item Total Effect (TE) : \[Y_{1,M_1}-Y_{0,M_0}\]
\item Indirect Effect (IE) ({\color{red} Mediation Effect}) :
  \[Y_{1,M_1} - Y_{1,M_0}\]
\item Causal effect of the change in $M$ on $Y$ while holding the
  treatment constant.
\item Represents the mechanism through $M$.
\item Contains unobservable {\color{red} $Y_{1,M_0}$}! 
\end{itemize}
\end{frame}

\begin{frame}
\begin{itemize}
\item Direct Effect (DE) :
\begin{eqnarray*}
\lefteqn{TE - IE }\\
& = & Y_{1,M_1}-Y_{0,M_0}
-\{Y_{1,M_1}-Y_{1,M_0}\}\\
 & = & Y_{1,M_0}-Y_{0,M_0}
\end{eqnarray*}
\item Causal effect of treatment $z$ on outcome $Y$ while holding
  mediator constant at its natural value under control.
\medskip
\item Represents causal mechanism around $M$.
\medskip
\item Also, contains unobservable  {\color{red} $Y_{1,M_0}$}! 
\end{itemize}
\end{frame}

\begin{frame}
\frametitle{Controlled Effect vs Natural Effect (Pearl, 2001)}
{\bf Controlled Direct Effect}
\begin{itemize}
\item $Y_{1,m}-Y_{0,m}$ for a certain level $m \in M$.
\item Causal effect if one can directly manipulate the mediator.
\end{itemize} 
{\bf Natural Direct Effect}
\begin{itemize}
\item $Y_{1,M_0}-Y_{0,M_0}$.
\item Mediator is set to have its natural value.
\end{itemize}
{\bf Average Causal Effect of Intervention}
\begin{itemize}
\item Since we observe either $Y_{1,M_1}$ and $Y_{0,M_0}$ for
  an individual, compute $E\{Y_{1,M_1}-Y_{0,M_0}\}$.
\end{itemize}
\end{frame}

\begin{frame}
\frametitle{Assumptions for Identification}
Several combinations of assumptions are typically required for inference on causal effects.
\begin{itemize}
\item Sequential ignorability 
\[\{Y_{z',m},M_z\} \perp Z \,| \,X\]
\[Y_{z',m} \perp M_z \,| \,Z, X\]
\item No-interaction assumption (about controlled direct effect)  
\[Y_{1,m}-Y_{0,m} = B\]
\item Exclusion restriction
\[Y_{z,m}=Y_{z',m} \hspace{2mm} \mbox{ for all } z, z' \mbox{ and for all } m\] states treatments affect potential outcomes
only through mediators. So, no D.E.
\end{itemize}
\end{frame}

\begin{frame}
\frametitle{Literature}
\begin{enumerate}
\item {\color{red} Structural Equation Models} (SEM) (Baron \& Kenny, 1986; MacKinnon
  et al, 2002; Sobel, 1982; Preacher \& Hayes, 2008)

: highly parametric. assume sequential igonorability and no interaction.
\item {\color{red} Principal Stratification} (PS) (Frangakis \& Rubin,
  2002, Gallop et al, 2009)

: assume SUTVA. problems with drawing the population effects. (VanderWeele, 2011)
\item {\color{red} Instrumental Variables} (IV) (Angrist et al., 1996;
  Albert, 2008; Sobel, 2008)

: assume exclusion restriction
\item {\color{red} Structural Mean Models} (SMM) (Ten Have et al.,
  2004)

: assume no-interaction.
\item {\color{red} Marginal Structural Models} (MSM) (Robins et al., 2000)

: assume sequential ignorability. 
\end{enumerate}
\end{frame}


\begin{frame}
\frametitle{Goals of the Dissertation}
Present a Bayesian framework for inference on causal effects of
mediation
\bigskip
\begin{enumerate}
\item Use different and intuitive assumptions with sensitivity parameters.
\bigskip
\item Examine sensitivity to the assumptions.
\bigskip
\item Develop for several common scenarios (e.g.,single mediator,
  longitudinal mediation, multiple mediators).
\bigskip
\end{enumerate}
\end{frame}


\section{Single Mediator Setting}
\begin{frame}

\center {\bf \large Single Mediator Case}
 
\center{assume only one mediator on the causal pathway}

\end{frame}

\begin{frame}
\frametitle{Review of Notation}
From the potential outcome framework, define
\medskip
\begin{itemize}
\item Binary Treatment : $Z$ ( Control=0 vs. Intervention=1).
\medskip
\item Potential Mediator : $M_z$ under treatment $z$. 
\medskip
\item Observed Mediator : $M_{obs} = Z M_1 + (1-Z) M_0$
\medskip
\item Potential Outcome : $Y_{z,M_{z'}}$ for $z,z' \in \{0,1\}$
\medskip
\item Observed Outcome : $Y_{obs} = Z Y_{1,M_1} + (1-Z) Y_{0,M_0}$
\end{itemize}
\medskip
Note that $Y_{1,M_0}$ and $Y_{0,M_1}$ are not observable.
\end{frame}

\begin{frame}
\frametitle{Assumptions}
\begin{block}{Assumption 1 (Randomization)}
\[\{Y_{z',m}, M_z \} \perp Z.\]
\end{block}
\begin{block}{Assumption 2a }
Let $D=(M_1-M_0)$. For a fixed $\epsilon$,
$$P(Y_{1,M_{0}} = 1 | M_{0}=m, |D| < \epsilon) = P(Y_{1,M_{1}} = 1 | M_1=m, |D| < \epsilon).$$
\end{block}
\begin{block}{Assumption 2b }
For a fixed $\epsilon$, and $\chi$, 
we assume 
\begin{eqnarray*}
 \lefteqn{P(Y_{1,M_{0}} = 1 | M_{0}=m, |D| \geq \epsilon) =}\\
 && \chi^{\mbox{sgn}(d)}P(Y_{1,M_{1}} = 1 | M_1=m, |D| \geq \epsilon).
\end{eqnarray*}
\end{block}
\end{frame}
 



\begin{frame}
\begin{block}{Assumption 3}
$$f_{M_{z'}}(m_{z'} | m_{z}, Y_{z,M_z}) = 
f_{M_{z'}}(m_{z'} | m_{z})$$
\end{block}
\begin{block}{Assumption 4 (Joint Distribution of Mediators)}
 This follows a Gaussian copula model (Nelsen, 1999),
\[
F_{M_0, M_1}(m_0,
m_1)=\Phi_2\left[\Phi_1^{-1}\{F_{M_0}(m_0)\},
\Phi_1^{-1}\{F_{M_1}(m_1)\}\right]
\]
\noindent where $\Phi_1$ is the univariate standard normal CDF and $\Phi_2$
is the bivariate normal CDF with mean $(0,0)^T$, variance $(1,1)^T$
and correlation $\rho \in (-1,1)$.  
\end{block}
\begin{block}{Assumption 5 (Conditional Independence)}
 $$f_{(1,M_1),(1,M_0),(0,M_0)}(y_{11}, y_{10}, y_{00}
 |m_0, m_1) = \qquad\qquad$$
 $$\qquad f_{1,M_1}(y_{11}| m_0, m_1)f_{1,M_0}(y_{10} | m_0, m_1)f_{0,M_0}(y_{00} | m_0, m_1).$$
\end{block}
\end{frame}



\begin{frame}
\frametitle{Identification}
\begin{block}{Theorem}
The joint posterior distribution of NIE and
NDE is identified under Assumptions 1-5.
\end{block}


\end{frame}





\begin{frame}
\frametitle{TOURS : weight management trial (Perri et al., 2008)}
\begin{itemize}
\item Randomized trial to compare the effectiveness of
extended care programs of weight management.
\item After completing a standard
six month lifestyle modification program, participants were randomly
assigned to telephone counseling, face-to-face counseling or an
education control group.
\item {\color{red} Treatment} : Face to face (FTF) vs Education
  control (EC).
\item {\color{red} Mediator} : The number of days with self-monitoring
  records for food intake (0-350) during 6 to 18 months.
\item {\color{red} Binary outcome} : Among those that lost $\geq 5\%$ of weight by 6
    months, indicator of whether they maintained the $\geq 5\%$ weight
    loss from 6-18 months.
\end{itemize}
\end{frame}


%\begin{frame}

%\begin{table}[h]
%\caption{Posterior means and credible intervals :
%  $\rho$=0.3.}
%\centering
%\begin{tabular}{c c c c c}
%\hline
%$\epsilon$ & $\chi$ & NDE & NIE & TE \\ [0.5ex]
%\hline
%$50$ & 1 & 0.078  & 0.007 & 0.085 \\
% & & (-0.073,0.25) & (-0.086,0.12) & (-0.070,0.25)\\
%\hline
%$50$ & 1.3 & 0.087  & -0.0023 & 0.085 \\
% & & (-0.078,0.26) & (-0.10,0.10) & (-0.070,0.25)\\
%\hline
%\hline
%$75$ & 1 & 0.077  & 0.007 & 0.085 \\
% & & (-0.076,0.25) & (-0.095,0.12) & (-0.070,0.25)\\
%\hline
%$75$ & 1.3 & 0.086  & -0.001 & 0.085 \\
% & & (-0.079,0.26) & (-0.10,0.10) & (-0.070,0.25)\\
%\hline
%\hline
%$100$ & 1 & 0.078  & 0.006 & 0.085 \\
% & & (-0.071,0.25) & (-0.095,0.12) & (-0.070,0.25)\\
%\hline
%$100$ & 1.3 & 0.085  & 0.0001 & 0.085 \\
% & & (-0.079,0.26) & (-0.10,0.10) & (-0.070,0.25)\\
%\hline
%\end{tabular}
%\end{table}
%\end{frame}

\begin{frame}

\begin{table}[h]
\caption{Posterior means and credible intervals :
  $\rho$=0.7.}
\centering
\begin{tabular}{c c c c c}
\hline
$\epsilon$ & $\chi$ & NDE & NIE & TE \\ [0.5ex]
\hline
$50$ & 1 & 0.077  & 0.007 & 0.085 \\
 & & (-0.073,0.25) & (-0.092,0.13) & (-0.070,0.25)\\
\hline
$50$ & 1.3 & 0.088  & -0.003 & 0.085 \\
 & & (-0.085,0.26) & (-0.10,0.099) & (-0.070,0.25)\\
\hline
\hline
$75$ & 1 & 0.077  & 0.007 & 0.085 \\
 & & (-0.066,0.25) & (-0.088,0.12) & (-0.070,0.25)\\
\hline
$75$ & 1.3 & 0.086  & -0.001 & 0.085 \\
 & & (-0.087,0.26) & (-0.097,0.10) & (-0.070,0.25)\\
\hline
\hline
$100$ & 1 & 0.078  & 0.007 & 0.085 \\
 & & (-0.069,0.25) & (-0.091,0.12) & (-0.070,0.25)\\
\hline
$100$ & 1.3 & 0.084  & 0.0006 & 0.085 \\
 & & (-0.084,0.26) & (-0.10,0.10) & (-0.070,0.25)\\
\hline
\end{tabular}
\end{table}

\end{frame}


\begin{frame}
\frametitle{Summary}
\begin{itemize}
\item Based on analysis, the number of self-monitoring food records
  completed was not a mediator.
\end{itemize}
\end{frame}


\section{Longitudinal Mediator}

\begin{frame}

\center {\bf \large Longitudinal Mediator}
 
\center{assume single treatment and longitudinal outcomes and mediators }

\end{frame}

\begin{frame}
\frametitle{Approaches}
{\bf In the time-varying treatment setting}
\begin{itemize}
\item {\color{red} Regression based approach}:  Maxwell et al. (2011). Structural models between $Z_t$, $M_t$ and $Y_t$ at each time and over all time periods.
\item {\color{red} Marginal structural models}: van der Laan and Petersen (2004). Estimating the marginal mean of potential outcome $E(Y^{\bar{z},\bar{M}^{\bar{z'}}})$
where $\bar{z}$ denotes treatment histories.
\end{itemize}
{\bf Others}
\begin{itemize}
\item {\color{red} Parallel-process model}: Cheong et al. (2003). Mediators and outcomes are modeled as different parallel processes with latent factors. Specify structural equation models of latent factors.
\item {\color{red} Principal strata approach}: Lin et al. (2008). Use a mediator to form principal strata and draw principal strata direct and indirect effects. 
\end{itemize}
\end{frame}


\begin{frame}
\frametitle{Basic Framework}
\begin{figure}[h]
\centering
\scalebox{0.26}
{\includegraphics{figure4.png}}
\caption{Mediation model with time $t=1,2,3$.}
\begin{itemize}
\item Outcome $Y_t$ is affected by the preceding mediator $M_{t-1}$ and both have their own causal pathways.
\item Mediator $M_t$ is affected by the outcome $Y_t$.
\item Outcomes and mediators have direct paths from the treatment $Z$.
\end{itemize}
\end{figure}
\end{frame}

\begin{frame}
\frametitle{Longitudinal Mediation Analysis}
\begin{itemize}
\item Potential mediator : $M_t^z$
as the value of the mediator at time $t = 1, \cdots, T$
under treatment $Z=z$.
\item Full histories of mediators : $M^1 = (M_1^1,\cdots, M_T^1)$ and $M^0 =
(M_1^0,\cdots, M_T^0)$.
\item Potential outcome : $Y_t^{z,M_{t-1}^z}$
denotes the outcome that would be observed at time $t$ if $Z=z$.
\item Then, causal effects at time $t$, 
\begin{eqnarray*}
NDE_t & = & 
E(Y_t^{1,M^0_{t-1}}- Y_t^{0,M^0_{t-1}} )\\
 NIE_t & = &
E(Y_t^{1,M^1_{t-1}}- Y_t^{1,M^0_{t-1}} )\\
TE_t & = & NDE_t+NIE_t = E(Y_t^{1,M^1_{t-1}}- Y_t^{0,M^0_{t-1}} ).
\end{eqnarray*}
\item We update these by a Bayesian dynamic model and the observed data.
\end{itemize}

\end{frame}

\begin{frame}
\frametitle{Bayesian Dynamic Models (West \& Harrison, 1989)}
Define $V_t=(Y_t, M_{t-1})$ to be the
response at time $t$ and mediator at time $t-1$ assume a model
$p_t(V_t|\boldsymbol{\theta_t})$ parameterized by
$\boldsymbol{\theta_t}$.
\begin{block}{Modeling Assumption}
Conditional on a vector of
  state parameters, $\boldsymbol{\theta_t}$, $V_t=(Y_t,
M_{t-1})$ is independent of $V_{s-1}$ and $\boldsymbol{\theta_s}$
  for all values of $s<t$.
\end{block}
\begin{itemize}
\item Observation Model : $ V_t \sim
p_t(V_t | \boldsymbol{\theta_{t}}, V_{t-1})$.
\item Evolution Model : $(\boldsymbol{\theta_t}|\boldsymbol{\theta_{t-1}}) \sim p_e(\boldsymbol{\theta_t}|\boldsymbol{ \theta_{t-1}})$

for $t=1,\cdots, T.$
\end{itemize}
\end{frame}

\begin{frame}
\frametitle{Assumptions}
Let $D_{t-1}=M_{t-1}^1-M_{t-1}^0$. And let $y_t^{zz'}$ denote $y_t^{z,M^{z'}_{t-1}}$ for notational simplicity.
\begin{block}{Assumption 1a}
For a fixed $\epsilon$ at each time $t$,
\begin{eqnarray*}
\lefteqn{f_{1,M_{t-1}^0}(y_t^{10}|M_{t-1}^0=m, |D_{t-1}|\leq \epsilon, \boldsymbol{\theta_t})=}\\
& & \quad \quad \quad f_{1,M_{t-1}^1}(y_t^{11}|M_{t-1}^1=m, |D_{t-1}|\leq \epsilon, \boldsymbol{\theta_t}).
\end{eqnarray*}
\end{block}
\begin{itemize}
\item Among the subject for whom the treatment would have minimal
  impact on the mediator (quantified by $\epsilon$), the distributions
  of the outcomes are the same whether the mediator value induced by
  $z=1$ or $z=0$ conditional on $\boldsymbol{\theta_t}$.
\end{itemize}
\end{frame}

\begin{frame}
\begin{block}{Assumption 1b}
For a fixed $\epsilon$ and $\chi$,
\begin{eqnarray*}
\lefteqn{f_{1,M_{t-1}^0}(y_t^{10}|M_{t-1}^0=m, |D_{t-1}|\geq \epsilon, \boldsymbol{\theta_t})\propto}\\
& & \text{exp}(\log(\chi^{sgn(D_{t-1})}) y_t^{11}) f_{1,M_{t-1}^1}(y_t^{11}|M_{t-1}^1=m, |D_{t-1}|\geq \epsilon, \boldsymbol{\theta_t}).
\end{eqnarray*}
\end{block}
\begin{itemize}
\item The second assumption is for the subgroup of subjects for whom the intervention has a greater than $\epsilon$ effect on $M_{t-1}$.
\end{itemize}
Thus, Assumption 1a and b stratify the population into those where the intervention has a large versus small effect on the mediator.
\end{frame}

\begin{frame}
\begin{block}{Assumption 2}
For a fixed $z$ at each time $t$,
\[f_{z,M_{t-1}^z}(y_t^{zz} | m_{t-1}^z, m_{t-1}^{(1-z)}, \boldsymbol{\theta_t}) = f_{z,M_{t-1}^z}(y_t^{zz}|m_{t-1}^z, \boldsymbol{\theta_t}).\]
\end{block}
\begin{itemize}
\item The potential outcomes $Y_t^{z,M_{t-1}^z}$ are independent of the mediator under the other treatment $M_{t-1}^{(1-z)}$ conditional on the mediator associated with the potential outcomes $M_{t-1}^z$ and the vector of state parameters $\boldsymbol{\theta_t}$.
\end{itemize}
\end{frame}

\begin{frame}
\begin{block}{Assumption 3 (Conditional Independence)}
\begin{eqnarray*}
\lefteqn{f_{(1,M_{t-1}^1),(1,M_{t-1}^0),(0,M_{t-1}^0)}(y_t^{11},y_t^{10},y_t^{00}|m_{t-1}^0, m_{t-1}^1,\boldsymbol{\theta_t})=}\\
& & f_{1,M_{t-1}^1}(y_t^{11}|m_{t-1}^0, m_{t-1}^1,\boldsymbol{\theta_t}) \times
f_{1,M_{t-1}^0}(y_t^{10}|m_{t-1}^0, m_{t-1}^1,\boldsymbol{\theta_t})\\
& & \times
f_{0,M_{t-1}^0}(y_t^{00}|m_{t-1}^0, m_{t-1}^1,\boldsymbol{\theta_t}).
\end{eqnarray*}
\end{block}
\begin{itemize}
\item Not necessary to estimate $E[NIE_t|data]$ and $E[NDE_t|data]$.
\item Necessary to estimate other features of the posterior distribution of $NIE_t$ and $NDE_t$ such as an upper bound on the variance.
\end{itemize}
\end{frame}

\begin{frame}
\begin{block}{Assumption 4}
For the joint distribution of mediators, we assume a Gaussian copula model
\begin{eqnarray*}
\lefteqn{F_{M_t^0,M_t^1}(m_t^0,m_t^1 | \boldsymbol{\theta_{t+1}}) =}\\ & &\Phi_2[\Phi_1^{-1}\{F_{M_t^0}(m_t^0|\boldsymbol{\theta_{t+1}})\}, \Phi_1^{-1}\{F_{M_t^1}(m_t^1|\boldsymbol{\theta_{t+1}})\}]
\end{eqnarray*}
\end{block}
\begin{itemize}
\item $\phi_1$ is the univariate standard normal CDF and $\phi_2$ is the bivariate normal CDF with mean $(0,0)^T$, variance $(1,1)^T$ and correlation $\rho$.
\item $\rho$ is a sensitivity parameter since the two mediators are not observed at the same time.
\end{itemize}
\end{frame}


\begin{frame}
\frametitle{Identification}
\begin{block}{Theorem}
The joint posterior distribution of $NIE_t$ and $NDE_t$ for each $t = 1, \cdots, T$ is identified under Assumptions 1-4 and the Bayesian dynamic model.
\end{block}
\end{frame}

\begin{frame}
\frametitle{Specification of Models}
{\bf Observation model}
\begin{itemize}
\item $(Y_t^{zz},Y_{t-1}^{zz}) \sim \text{Mult}(\pi_{1t}^z,\pi_{2t}^z,\pi_{3t}^z) \quad \text{for} t=2, \cdots, T.$
\item For mediators, we need two conditional distributions,
\[f_{M_{t-2}^z}(m_{t-2}^z|y_t^{zz}, y_{t-1}^{zz}) \text{ and } f_{M_{t-1}^z}(m_{t-1}^z|m_{t-2}^z,y_t^{zz}, y_{t-1}^{zz}).\]
\item We specify conjugate normal-normal MDP models (MacEachern and M\"uller, 1998) for these.
\end{itemize}
{\bf Evolution model}
\begin{itemize}
\item $\boldsymbol{\pi_t}^z | \boldsymbol{\pi_{t-1}}^z \sim \text{Dir}(\delta_t^z \boldsymbol{\pi_{t-1}}^z),$
where the hyperparameter $\delta_t$ is given a uniform shrinkage prior (Daniels, 1999).
\item For the mediator, we update parameters of the base measures of the MDP models (Observation models).
\end{itemize}
\end{frame}



\begin{frame}
\frametitle{Application: CTQ II (Marcus et al., 2005)}
\begin{itemize}
\item A randomized clinical trial to study the effect of moderate-intensity exercise on smoking cessation over 8 weeks.
\item Moderate intensity exercise intervention $(n_1=109)$ vs. equivalent staff contact time control $(n_0=108)$.
\item Potential {\color{red} mediator} at week $t$ is the difference between baseline weight and weight measured at week $t$.
\item {\color{red} Binary outcome} at week $t$ is quit status (quit=1).
\item Since the target quit week was week 3, we assess the effects from week 4 to week 8.
\item Assume missingness is ignorable.
\end{itemize}
\end{frame}

\begin{frame}
\frametitle{Sensitivity Parameters}
\begin{itemize}
\item The difference in the average weight changes between two groups at week 8 was 0.7. with S.D. of 1.65.
\item Consider differences more than 0.7 as the case that Assumption
  1a does not hold. 
\item $\epsilon$ varies from 0.7 to 2.3 (average+ 1 S.D.).
\item We assume the impact of treatment on the mediator being more than $1\sim 2$ pounds could reflect a negative impact up to the odds ratio of about 0.5.
\item $\chi$ varies from 0.5 to 1.
\item We assume the positive correlation between mediators. $\rho\in [0,1)$.
\end{itemize}
\end{frame}

\begin{frame}
\begin{figure}[h]
\centering
\scalebox{0.44}
{\includegraphics{5.pdf}}
\caption{Cessation rates of the exercise intervention and the contact
  control under ignorability}
\end{figure}
\end{frame}

\begin{frame}
\begin{figure}[h]
\centering
\scalebox{0.22}
{\includegraphics{2.pdf}}
\caption{NIE, NDE and TE for week 4,5,6,7,8 when $\rho=0.3$}
\end{figure}
\end{frame}


\begin{frame}
\begin{figure}[h]
\centering
\scalebox{0.22}
{\includegraphics{3.pdf}}
\caption{NIE, NDE and TE for week 4,5,6,7,8 when $\rho=0.5$}
\end{figure}
\end{frame}


\begin{frame}
\begin{figure}[h]
\centering
\scalebox{0.22}
{\includegraphics{4.pdf}}
\caption{NIE, NDE and TE for week 4,5,6,7,8 when $\rho=0.7$}
\end{figure}
\end{frame}

\begin{frame}
\frametitle{Discussion}
\begin{itemize}
\item Based on analysis, the effect of moderate exercise intervention
  vs. the staff contact control was marginally significant (in a
  negative way).
\item The longitudinal mediator, weight change, had a significant
  positive impact
\item For future work,
\begin{itemize}
\item Explore time-varying treatments instead of a single time treatment.
\item Also, incorporate time-varying covariates to weaken some of assumptions.
\end{itemize}
\end{itemize}
\end{frame}


\section{Multiple Mediators}


\begin{frame}

\center {\bf \large Multiple Mediators}
 
\center{assume multiple mediators on the causal pathway }

\end{frame}


\begin{frame}
\frametitle{Setting of Multiple Mediators}
\begin{itemize}
\item We can extend the previous method to the case of multiple
  mediators.
\begin{eqnarray*}
\begin{array}{ccccc}
Z & & \longrightarrow & & Y \\
  & \searrow & \vdots & \nearrow & \\
  &          & M^1, \cdots, M^n & &
\end{array}     
\end{eqnarray*} 
\item We focus on the situation that mediators are measured at the
  same time (not sequentially).
\end{itemize}
\end{frame}

\begin{frame}
\frametitle{Framework for Multiple Mediators}
\begin{itemize}
\item $M_z^k$: $k$-th potential mediator under treatment $Z=z$.
\item $Y_{z,M_{z_1}^1M_{z_2}^2\cdots ,M_{z_K}^K}$: Potential outcome
  under randomization to intervention level $z$ and realized mediator
  values $M_{z_1}^1M_{z_2}^2\cdots ,M_{z_K}^K$ where $\{z, z_1,
  \cdots, z_K\} \in \{0, 1\}^{\otimes (K+1)}$.
\item Observed mediator: $M^k=ZM_1^k+(1-Z)M_0^k$.
\item Observed outcome:
  $Y=ZY_{1,M_1^1M_1^2\cdots M_1^K}+(1-Z)Y_{0,M_0^1M_0^2\cdots M_0^K}.$
\item Let $\mathbf{M}_z = \{M_z^1,M_z^2,\cdots,M_z^K\}$ for notational
  simplicity.
\item The potential outcome $Y_{z,M_z^1M_z^2\cdots M_z^K}$ can be
  re-written as $Y_{z,\mathbf{M}_z}$.
\end{itemize}
\end{frame}

\begin{frame}
\frametitle{Decomposition of the Total Effect}
\begin{itemize}
\item Total Effect (TE) : 
\[TE  = E(Y_{1,\mathbf{M}_1}-Y_{0,\mathbf{M}_0}).\]
\item Natural Direct Effect (NDE) : 
\[ NDE = E(Y_{1,\mathbf{M}_0}-Y_{0,\mathbf{M}_0}).\]
\item {\color{red} Joint Natural Indirect Effect (JNIE)} :
\[JNIE= TE - NDE = E(Y_{1,\mathbf{M}_1}-Y_{1,\mathbf{M}_0})\]
which is  the aggregate effect of all mediators.
\end{itemize}
Also, decompose $JNIE$ further to get mediator-specific indirect effects (e.g., $NIE_1, \cdots, NIE_K$).
\end{frame}

\begin{frame}
\frametitle{Decomposition of JNIE}
WLOG, assume 3 mediators on the causal pathway,
\begin{figure}[h]
\centering
\scalebox{0.53}
{\includegraphics{decomp.png}}
\caption{Partitioning of the $JNIE$}
\end{figure}
\end{frame}

\begin{frame}

{\bf Mediator-specific indirect effect ($NIE_k$) is}
\begin{itemize}
\item $NIE_1 = E(Y_{1,\mathbf{M}_1} - Y_{1,M_0^1M_1^2M_1^3}).$
\end{itemize} 
\smallskip

{\bf Simultaneous indirect effect of two mediators ($SE_{jk}$) is}
\begin{itemize}
\item 
$SNIE_{12}  =  E(Y_{1,\mathbf{M}_1} - Y_{1,M_0^1M_1^2M_1^3} - Y_{1,M_1^1M_0^2M_1^3} + Y_{1,M_0^1 M_0^2M_1^3}).$
\end{itemize}
\smallskip

{\bf Simultaneous indirect effect of three mediators ($SE_{123}$) is}
\begin{itemize}
\item 
$SNIE_{123}  =  E(Y_{1,\mathbf{M}_1}-Y_{1,M_0^1M_1^2M_1^3}-Y_{1,M_1^1M_0^2M_1^3}-Y_{1,M_1^1M_1^2M_0^3}+Y_{1,M_0^1 M_0^2M_1^3}+Y_{1,M_0^1M_1^2 M_0^3}+Y_{1,M_1^1M_0^2 M_0^3}-Y_{1,\mathbf{M}_0}).$

\end{itemize}
\bigskip

Then, 
\begin{eqnarray*}
JNIE & = & \sum_k NIE_k -\sum_{j,k:j \neq k} SE_{jk} + SE_{123} 
 =  E(Y_{1,\mathbf{M}_1}-Y_{1,\mathbf{M}_0})\\
& = & {\color{red} TE -NDE}.
\end{eqnarray*}
\end{frame}


\begin{frame}
\frametitle{Assumptions}
All assumptions are presented in the presence of $K=3$ mediators.
\begin{block}{Assumption1}
For a given mediators under intervention $z=1$, the conditional
distributions of the outcome are the same whether those mediator values
were induced by $z=1$ or $z=0$.
\end{block}
\begin{itemize}
\item
 \resizebox{0.9\hsize}{!}{ $f_{1,M_0^1M_0^2M_0^3}(y_{1,\mathbf{M}_0}|\mathbf{M}_0=\mathbf{m}_0,
  \mathbf{M}_1)=f_{1,M_1^1M_1^2M_1^3}(y_{1,\mathbf{M}_1}|\mathbf{M}_0,
  \mathbf{M}_1=\mathbf{m}_0).$}
Other cases are in page \ref{app1}.
\end{itemize}
\begin{block}{Assumption 2}
\[f_{z,M_z^1M_z^2M_z^3}(y_{z,\mathbf{M}_z} | \mathbf{m}_z,\mathbf{m}_{1-z}
)=f_{z,M_z^1M_z^2M_z^3}(y_{z,\mathbf{M}_z} | \mathbf{m}_z
).\]
\end{block}
\end{frame}

\begin{frame}
\begin{block}{Assumption 3 (Joint Distribution of Mediators)}
\begin{eqnarray*}
\lefteqn{F_{\mathbf{M}_0,\mathbf{M}_1}(\mathbf{m}_0,\mathbf{m}_0)=}\\
& &\Phi_6 [\Phi_1^{-1}\{F_{M_0^1}(m_0^1)\},\Phi_1^{-1}\{F_{M_0^2}(m_0^2)\},\Phi_1^{-1}\{F_{M_0^3}(m_0^3)\},\\
& & \qquad \Phi_1^{-1}\{F_{M_1^1}(m_1^1)\},\Phi_1^{-1}\{F_{M_1^2}(m_1^2)\},\Phi_1^{-1}\{F_{M_1^3}(m_1^3)].
\end{eqnarray*}
\end{block}
\begin{block}{Assumption 4 (Conditional Independence)}
All potential outcomes that define JNIE, $\text{NIE}_k$, $\text{SNIE}_{jk}$, etc. are conditionally independent given all potential mediators under $z=0,1$.
\end{block}

\end{frame}



\begin{frame}
\frametitle{Identification}
\begin{block}{Theorem}
The joint posterior distribution of NDE, JNIE, $\text{NIE}_k$, $\text{SNIE}_{jk}$ and $\text{SNIE}_{123}$ is identified under Assumption 1-4. 
\end{block}
\end{frame}



%\begin{frame}
%\frametitle{Application: STRIDE trial (Marcus et al., 2007)}
%\begin{itemize}
%\item Evaluate the efficacy of interventions targeting physical activity adoption and maintenance.
%\item Randomized to 3 treatment arms: telephone-based, print-based interventions and contact-contact group.
%\item Combine two interventions into a single intervention ($n_1=161$ vs. $n_0=78$).
%\item At baseline, 6 and 12 months, participants' physical activity logs were reported by mail.
%\item {\color{red}Binary outcome} has 1 if more than 150 minutes per week of moderate intensity activity at month 12.
%\item Three {\color{red}potential mediators} are self-efficacy ($M^1$), behavioral processes ($M^2$) and cognitive processes ($M^3$) which are measured by questionnaires.
%\end{itemize}
%\end{frame}


%\begin{frame}
%To make the covariance matrix positive definite in terms of the previous restriction and data, $\rho_1$ and $\rho_2$ are free to vary in the area,
%\begin{figure}[h]
%\centering
%\scalebox{0.28}
%{\includegraphics{corr.png}}
%\caption{Possible ranges of $\rho_1$ and $\rho_2$.}
%\end{figure}
%Thus, we consider (1) a uniform prior on the region; (2) $\rho_1=0.3$ and $\rho_%2=0.15$; (3) $\rho_1=0.6$ and $\rho_2=0.45$.
%\end{frame} 

%\begin{frame}


%\begin{table}[h]
%\caption{Estimates and 95\% credible intervals (parentheses) of $TE$, $JNIE$ and %$NDE$ for  (1) a uniform prior on the region; (2) $\rho_1=0.3$ and $\rho_2=0.15$;% (3) $\rho_1=0.6$ and $\rho_2=0.45$}\label{out1}
%\centering
%\begin{tabular}{ c c c c }
%\hline
 % Case & $TE$ &  $JNIE$  & $NDE$ \\
%\hline
%\multirow{2}{*}{(1)} & 0.179 & 0.106 & 0.073 \\
%& (0.06, 0.28) & (0.06, 0.15) & (-0.03, 0.17) \\ 
%\hline
%\multirow{2}{*}{(2)}  & 0.179 & 0.106 & 0.073 \\
 % & (0.06, 0.28) & (0.06, 0.15) & (-0.03, 0.18) \\ 
%\hline
%\multirow{2}{*}{(3)}& 0.179 & 0.106 & 0.073 \\
%& (0.06, 0.28) & (0.06, 0.15) & (-0.04, 0.18) \\ 
%\hline\\
%\end{tabular}
%\end{table}
%\end{frame}

%\begin{frame}
%\begin{table}[h]
%\caption{Estimates and 95\% credible intervals (parentheses) of $NIE$'s for  (1) a u%niform prior on the region; (2) $\rho_1=0.3$ and $\rho_2=0.15$; (3) $\rho_1=0.6%$ and $\rho_2=0.45$}\label{out2}
%\centering
%\begin{tabular}{ c c c c }
%\hline
%  Case & $NIE_1$ &  $NIE_2$  & $NIE_3$ \\
%\hline
%\multirow{2}{*}{(1)} & 0.032 & 0.121 & -0.035 \\
%& (0.01, 0.05) & (0.07, 0.16) & (-0.06, -0.01) \\ 
%\hline
%\multirow{2}{*}{(2)}  & 0.032 & 0.112 & -0.041 \\
 % & (0.01, 0.05) & (0.07, 0.15) & (-0.07, -0.01) \\ 
%\hline
%\multirow{2}{*}{(3)}& 0.031 & 0.119 & -0.036 \\
%& (0.01, 0.06) & (0.07, 0.16) & (-0.06, -0.01) \\ 
%\hline\\
%\end{tabular}
%\end{table}
%\end{frame}

%\begin{frame}
%\begin{table}[h]
%\caption{Estimates and 95\% credible intervals (parentheses) of $SNIE$'s for  (1) a %uniform prior on the region; (2) $\rho_1=0.3$ and $\rho_2=0.15$; (3) $\rho_1=0.%6$ and $\rho_2=0.45$}\label{out3}
%\centering
%\begin{tabular}{ c c c c c}
%\hline
%  Case & $SNIE_{12}$ &  $SNIE_{13}$  & $SNIE_{23}$ & $SNIE_{123}$ \\
%\hline
%\multirow{2}{*}{(1)} & 0.016 & 0.012  & -0.005 & 0.011\\
%& (-0.008, 0.05) & (-0.02, 0.03) & (-0.04, 0.01) & (-0.03, 0.04)\\ 
%\hline
%\multirow{2}{*}{(2)} & 0.018 & 0.010  & -0.019 & 0.011\\
 % & (-0.004, 0.05) & (-0.01, 0.03) & (-0.05, 0.004) & (-0.02, 0.04)\\ 
%\hline
%\multirow{2}{*}{(3)} & 0.017 & 0.009  & -0.008 & 0.011\\
%& (-0.007, 0.04) & (-0.01, 0.03) & (-0.04, 0.02) & (-0.02, 0.04)\\ 
%\hline
%\end{tabular}
%\end{table}
%\end{frame}

%\begin{frame}
%\frametitle{Discussion}

%\end{frame}

\begin{frame}

\center{Thank You.}

\end{frame}


\begin{frame}\label{app1}
\frametitle{Appendix}
\begin{block}{Assumption1}
For a given mediators under intervention $z=1$, the conditional
distributions of the outcome are the same whether those mediator values
were induced by $z=1$ or $z=0$.
\end{block}
\begin{itemize}
\item
  \resizebox{0.9\hsize}{!}{
    $f_{1,M_0^1,M_1^2,M_0^3}(y_{1,M_0^1,M_1^2,M_0^3}|M_0^1=m^1,M_0^2,M_0^3,M_1^1,M_1^2=m^2,M_1^3=m^3)$}

\resizebox{0.8\hsize}{!}{
  $=f_{1,M_1^1,M_1^2,M_1^3}(y_{1,\mathbf{M}_1}|\mathbf{M}_0,M_1^1=m^1,M_1^2=m^2,M_1^3=m^3).$}
\item   \resizebox{0.9\hsize}{!}{
    $f_{1,M_1^1,M_1^2,M_0^3}(y_{1,M_1^1,M_1^2,M_0^3}|M_0^1,M_0^2,M_0^3=m^3,M_1^1=m^1,M_1^2=m^2,M_1^3)$}

\resizebox{0.8\hsize}{!}{
  $=f_{1,M_1^1,M_1^2,M_1^3}(y_{1,\mathbf{M}_1}|\mathbf{M}_0,M_1^1=m^1,M_1^2=m^2,M_1^3=m^3).$}
\end{itemize}
\end{frame}

\end{document}